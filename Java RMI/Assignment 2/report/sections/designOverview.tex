\section{Design overview}
The main components of the \texttt{CarRental}'s distributed design are located in the \texttt{session} package. It comprises the three remotely accessible interfaces: \texttt{SessionManagerI}, \texttt{RentalSessionI}, and \texttt{ManagerSessionI}. The first manages the association between renters and their rental session. A client may create a new rental session only via this manager. Clients are identified by their name, which is supposed to be unique. This class maintain a conversational state with the corresponding client. This is the reason why it is a stateful session. Lastly, \texttt{ManagerSessionI}, provides a stateless session for managing purposes. It is stateless because there is no need for conversational state. The intended behavior of these components is as follows. The renter asks the session manager for a unique rental session. This rental session is made persistent and returned to the client. From now on a binding $<$client, session$>$ is created and maintained until the client terminates the session. %TODO What if the client doesn't terminate?
A manager session, on the other hand, can be open on the fly without the need of the aforementioned binding. A manager uses this session for managing purposes without storing any data. 

The \texttt{rental} package implements the application's business logic. There is only one addition to the provided code: \texttt{RentalStore}. This class implements the naming service and provides company registration, unregistration and lookup.